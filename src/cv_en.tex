%----------------------------------------------------------------------------------------
%	PACKAGES AND OTHER DOCUMENT CONFIGURATIONS
%----------------------------------------------------------------------------------------

\documentclass[a4paper]{cvtemplate_en} % a4paper/letterpaper

\graphicspath{{img/}}

\addbibresource{publications.bib}

% Education
\education{

\educationItem
	{2020-2023}
	{PhD in Physics}
	{}
	{Universit\'e de Montpellier~\worldflag{FR}}
	{Under the joint supervision of \mbox{Ludovic Berthier} and \mbox{Robert L. Jack.}}

\educationItem
	{2016-2018}
	{MSc in Physics}
	{}
	{\'Ecole normale sup\'erieure de Lyon~\worldflag{FR}}
	{Specialisation in computational physics, soft matter, and statistical physics.}

\educationItem
	{2015-2016}
	{BSc in Physics}
	{}
	{\'Ecole normale sup\'erieure de Lyon~\worldflag{FR}}
	{}

\educationItem
	{2013-2015}
	{Classes pr\'eparatoires aux grandes \'ecoles (PCSI/PC\textsuperscript{*})}
	{}
	{Lyc\'ee Lakanal, Sceaux~\worldflag{FR}}
	{%Ranked 59\textsuperscript{th} in the \'Ecole normale sup\'erieure de Lyon entrance exam.
    }

\vspace{-8pt}
{\hfill\rule{0.5\textwidth}{1pt}\hfill}
\vspace{8pt}

\educationItem
	{2018-2019 (Gap year)}
	{MA in Social sciences}
	{1\textsuperscript{st} year}
	{\'Ecole normale sup\'erieure de Lyon~\worldflag{FR}}
	{}

}

% Programming skill bars
%\programming{{SageMath\slash Mathematica / 3}, {Bash $\textbullet$ C\slash C++ / 5}, {Python / 5.5}}{6}
% Programming
\programming{
Python \faPython, C/C++, GNU/Linux \faLinux, Git, symbolic computation (SageMath), Julia, HTML/JS/CSS, \LaTeX.\\

\vspace{-15pt}
Molecular dynamics, parallelisation on CPU and GPU (OpenMP, HOOMD), biased path ensemble algorithms.
}

% Languages
\languages{
Fran\c{c}ais (French) -- Native speaker\hfill\oldworldflag{FR}\\
English -- Fluent\hfill\oldworldflag{GB}\\
Nederlands (Dutch) -- Beginner\hfill\oldworldflag{NL}\\
\vspace{-\baselineskip}
}

% Interests
\interests{
\begin{itemize}
\setlength\itemsep{-0.8pt}
\item[$\ast$] Extreme music.
\item[$\ast$] Free software, open knowledge.
\item[$\ast$] Environment protection.
\item[$\ast$] French and non-French literature.
\end{itemize}
}

%----------------------------------------------------------------------------------------
%	 PERSONAL INFORMATION
%----------------------------------------------------------------------------------------
% If you don't need one or more of the below, just remove the content leaving the command, e.g. \cvnumberphone{}

\cvname{YANN-EDWIN KETA} % Your name
\cvjobtitle{Associate professor of physics} % Job
% title/career

\cvbirthday{November 1995}
\cvlocation{Campus P. \& M. Curie (Jussieu)\\[-12.5pt] 7 quai Saint-Bernard, Paris 5\textsuperscript{e}}
\cvsite{yketa.xyz} % Website
\cvgithub{yketa} % GitHub profile
\cvorcid{0000-0001-7736-3676} % Orcid
\cvmail{yann-edwin.keta@espci.fr} % Email address

%----------------------------------------------------------------------------------------

\begin{document}

\makeprofile % Print the sidebar

%----------------------------------------------------------------------------------------
%	 RESEARCH
%----------------------------------------------------------------------------------------

\vspace{2mm}
\section{Research}

\begin{cvbody}

\cvitem
	{Sep 2025}
	{Present}
  {Associate professor (Ma\^itre de conf\'erences)}
  {UFR de Physique, Facult\'e des Sciences et Ing\'enierie,\\ Sorbonne Universit\'e \worldflag{FR}\\ Laboratoire de Physique et M\'ecanique des Milieux H\'et\'erog\`enes (PMMH), UMR 7636 CNRS, ESPCI Paris -- PSL, Sorbonne Universit\'e, Universit\'e Paris-Cit\'e \worldflag{FR}}
  {sorbonne.pdf,pmmh.pdf}{4.5}{-7}
	{}
  {\\[-20pt]}

\cvitem
	{Oct 2023}
	{Aug 2025}
  {Postdoc: ``Physical models of cell sheets''}
  {Instituut-Lorentz for Theoretical Physics,\\ Universiteit Leiden \worldflag{NL}}
  {leiden.pdf}{5.5}{-7}
	{Silke Henkes}
  {\\[15pt]}

\cvitem
	{Sep 2020}
	{Sep 2023}
  {\href{https://theses.hal.science/tel-04530690v1}{PhD: ``Emergence of disordered collective motion in dense systems of isotropic self-propelled particles''}}
  {Laboratoire Charles Coulomb (L2C), UMR 5221 CNRS,\\ Universit\'e de Montpellier \worldflag{FR}\\
  \href{https://scglass.uchicago.edu}{Simons collaboration on \textit{Cracking the Glass Problem}}}
  {umontpellier.pdf}{4.8}{-6.5}
	{Ludovic Berthier (Montpellier),\\ Robert L. Jack (Cambridge)}
  {\\}

\end{cvbody}

\section{ENS-funded internships}

\begin{cvbody}

\cvitem
	{Oct 2019}
	{July 2020}
  {``Large deviations of active particles''}
  {Department of Applied Mathematics and Theoretical Physics,\\ University of Cambridge \worldflag{GB}\\
	Laboratoire Mati\`ere et Syst\`emes Complexes (MSC),\\ UMR 7057 CNRS, Universit\'e Paris-Cit\'e \worldflag{FR}}
  {cambridge.pdf,uparis.pdf}{4.8}{-6.5}
	{Robert L. Jack, Michael E. Cates (Cambridge),\\ Fr\'ed\'eric van Wijland (Paris)}
  {\\}

\cvitem
	{Jan 2018}
	{Jul 2018}
  {``Glassy behaviour in phase-separating active matter''}
  {Stewart Blusson Quantum Matter Institute,\\ University of British Columbia \worldflag{CA}}
  {ubc.pdf}{}{-7}
  {J\"org Rottler}
  {\\
%\begin{itemize}
  % \item Implementation and analysis of a model system of polydisperse active Brownian disks with purely repulsive harmonic potential.\\
  % \item Implementation of the model in Python with the \href{https://glotzerlab.engin.umich.edu/hoomd-blue/}{HOOMD-blue} simulation toolkit.
  % \item Characterisation of the motility-induced phase separation and of the long-range correlated particles' motion.
%  \end{itemize}
  }

\cvitem
	{May 2017}
	{Jul 2017}
	{``Jamming criticality of spheroids''}
	{Institutionen f\"{o}r fysik, Ume\r{a} universitet \worldflag{SE}}
	{umea.pdf}{}{-7.5}
	{Peter Olsson}
	{\\
%\begin{itemize}
%	\item Modification of an already existing 2D circular frictionless granular particles dynamics C program in order to study 3D spheroidal frictionless particles.\\
% 	\item Exploitation of the numerical data in order to compare our results to the existing literature
% and identify unexpected and/or surprising phenomena.\\
%	\end{itemize}
	}

% \cvitem
% 	{Jun 2016}
% 	{Jul 2016}
% 	{``Leidenfrost drop impacts on surfaces with defects''}
% 	{Institut Lumi\`ere Mati\`ere, UMR 5306 CNRS,\\
% 	Universit\'e Claude Bernard Lyon 1 \worldflag{FR}}
% 	{ucbl}{}{-9}
% 	{Quentin Ehlinger, Christophe Ybert}
%  	{\\
% %	\begin{itemize}
% % 	\item Set-up and realisation of an experiment of drop fall on a superheated surface.
% % 	\item Development of numerical methods in Python and Matlab to compare our models to
% % our experimental results.
% % 	\end{itemize}
%  	}

\end{cvbody}

%----------------------------------------------------------------------------------------
%	 PUBLICATIONS
%----------------------------------------------------------------------------------------

\section{Publications}

\vspace{-4pt}
\begin{cvbody}
\parbox[t]{\textwidth}{
\begin{refsection}
\nocite{kammeraat2025correlated,keta2025longrange,naik2025keratins,keta2024emerging,ADD,dis_cm}
\setlength\bibitemsep{-5pt}
\printbibliography[heading=empty]
\end{refsection}
}
\end{cvbody}

\clearpage

\vspace{-30pt}
\begin{cvbody}
\parbox[t]{\textwidth}{
\begin{refsection}
\nocite{DAMTP2020,Umea2020,Umea2019,UBC2019}
\setlength\bibitemsep{-5pt}
\printbibliography[heading=empty]
\end{refsection}
}
\end{cvbody}
\vspace{5pt}

%----------------------------------------------------------------------------------------
%	 REFEREEING
%----------------------------------------------------------------------------------------

\section{Refereeing}

Communications Physics, Nature Communications, Nature Physics, Physical Review E,  Scientific Reports, SciPost, Soft Matter.
\vspace{25pt}

%----------------------------------------------------------------------------------------
%	 RESPONSIBILITIES
%----------------------------------------------------------------------------------------

\section{Responsibilities}

\begin{cvbody}

\cvitem
    {2025}
    {2026}
    {\href{https://intcha26.sciencesconf.org}{Co-organisation of the IntCha26 conference}}
    {Institut d'\'Etudes Scientifiques de Carg\`ese \worldflag{FR}}
    {}{}{}
    {}
    {The conference ``Interdisciplinary Challenges in Non-Equilibrium Physics'' aims to connect young researchers from diverse backgrounds at the interface between the physics of complex systems and biology.\vspace{5pt}}

\cvitem
	{2023}
	{2025}
	{\href{https://slam-leiden.nl}{Co-organisation of the ``Smart, Living, and Active Matter'' seminar}}
	{Universiteit Leiden \worldflag{NL}}
	{}{}{}
    {}
	{Hosts international speakers between 1 and 4 times a month.\vspace{12.5pt}}

\end{cvbody}

%----------------------------------------------------------------------------------------
%	 TEACHING
%----------------------------------------------------------------------------------------

\section{Teaching}

\begin{cvbody}

\cvitem
    {2025}
    {}
    {``Mathematics'', ``Physics of continuous media'' (Bachelor)\\ ``Granular materials'' (Master)}
    {Sorbonne Universit\'e \worldflag{FR}}
    {}{}{}
    {}
    {Teaching assistant.\vspace{5pt}}

\cvitem
	{2025}
	{}
	{``Active Matter'' (Advanced Topics Masters Course)}
	{Universiteit Leiden \worldflag{NL}}
	{}{}{}
	{}
	{Lecturer. Part of the course "Advanced Topics in Theoretical Physics" from the Dutch Research School of Theoretical Physics.\vspace{5pt}}

\cvitem
	{2024}
	{}
	{``Statistical Physics'' (Masters)}
	{Universiteit Leiden \worldflag{NL}}
	{}{}{}
	{}
	{Teaching assistant.\vspace{5pt}}

\cvitem
	{2022}
	{}
	{``Physics for life sciences'', ``Python for sciences'' (Bachelor)}
	{Université de Montpellier \worldflag{FR}}
	{}{}{}
    {}
	{Teaching assistant.\vspace{5pt}}

\cvitem
	{2018-2019\\ \mbox{}\hfill 2016-2017}
	{}
	{Oral interrogator (Physics, Chemistry, Mathematics)}
	{Lyc\'ee du Parc, Institution des Chartreux, Lyc\'ee La Martinière Diderot (Lyon) \worldflag{FR}}
	{}{}{}
	{}
	{\vspace{5pt}
	% Weekly physics, chemistry, and mathematics oral interrogator in 1\textsuperscript{st}- and 2\textsuperscript{nd}-year \textit{classes pr\'eparatoires aux grandes \'ecoles} (undergraduate level).\\
	}

% \cvitem
% 	{Sep 2016}
% 	{Mar 2017}
% 	{Oral interrogator}
% 	{Lyc\'ee du Parc, Lyon \worldflag{FR}}
% 	{}{}{}
% 	{}
% 	{}
%	{Weekly physics and chemistry oral interrogator in 2nd-year \textit{classes pr\'eparatoires aux grandes \'ecoles} (undergraduate level).\\}

\cvitem
	{2015}
	{2017}
	{Volunteer tutor (Physics, Chemistry, Mathematics)}
	{ENSeigner association, \'Ecole normale sup\'erieure de Lyon \worldflag{FR}}
	{}{}{}
	{}
	{\vspace{5pt}
	% \begin{itemize}
	% 	\item Tutoring in physics, chemistry and mathematics for high school students of Lyon.
	% 	\item Participation to the operation "R\'evise ton bac \`a la BmL" to help students preparing for the \textit{baccalaur\'eat} (French high school diploma).
	% \end{itemize}
	}

\end{cvbody}

\end{document}
